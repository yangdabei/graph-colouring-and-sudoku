\documentclass[../main.tex]{subfiles}

\begin{document}    
    
    In this section, we will explain how the solutions of a Sudoku puzzle can be represented as the points in the vanishing locus of a polynomial in 81 variables. The unique solution of a well-posed Sudoku puzzle can be read off from the reduced Gr\"obner basis of that ideal. We can formulate the algebraic approach to solving a Sudoku puzzle as a graph colouring problem where the aim is to construct a 9-colouring of a particular graph given a partial solution. It is worth noting that since we can regard solving a Sudoku as a graph problem, graph theoretical methods are much more efficient at solving a Sudoku puzzle than using Gr\"obner bases. The general problem of solving a Sudoku puzzle on an $n^2\times n^2$ grid with $n\times n$ blocks is NP-complete and has been subsequently converted to various other NP-complete problems such as constraint satisfaction \citep{simonis2005sudoku}, integer programming \citep{bartlett2008integer}, boolean satifiability \citep{ist2006sudoku}, and the Hamiltonian cycle problem \citep{haythorpe2016reducing}.

    A \emph{Sudoku board} is a particular example of a Latin square. A \emph{Latin square} of order $n$ is an $n\times n$ square grid filled with $n$ distinct symbols such that no symbol appears more than once in each row and column. Typically, Sudoku boards are $9\times 9$ Latin squares filled with the integers 1 to 9 with the additional constraint that the numbers appears only once in each of the nine distinguished $3\times 3$ blocks. We say that a \emph{Sudoku puzzle} is a partial solution to a Sudoku board. A \emph{well-posed} Sudoku puzzle is one that uniquely determines the rest of the board. We take Sudoku puzzle to mean well-posed Sudoku puzzle in the remainder of the report. Figure \ref{board} is an example of a Sudoku puzzle and its corresponding Sudoku board. It has been conjectured that the minimum number of clues given in a Sudoku puzzle to solve the puzzle is 17 but has not been proven.

        \begin{figure}[h!]
            \centering
            \begin{sudoku-block}
                | | | | | | | | |9|.
                |9|4| | | | |8|3| |.
                | | | |9| | |6| |2|.
                | |1| |7| | | |9| |.
                | | | | | |2| |5| |.
                | | |7| | |6| | | |.
                | | | | | |1| | | |.
                |5|8|1| |2| | | | |.
                | |6| | | |8|4| | |.
            \end{sudoku-block}
            \hspace*{10pt}
            \begin{sudoku-block}
                |7|2|8|6|1|3|5|4|9|.
                |9|4|6|2|5|7|8|3|1|.
                |1|3|5|9|8|4|6|7|2|.
                |8|1|2|7|4|5|3|9|6|.
                |6|9|4|8|3|2|1|5|7|.
                |3|5|7|1|9|6|2|8|4|.
                |4|7|3|5|6|1|9|2|8|.
                |5|8|1|4|2|9|7|6|3|.
                |2|6|9|3|7|8|4|1|5|.
            \end{sudoku-block}
            \caption{Well-posed Sudoku puzzle and board}
            \label{board}
        \end{figure}

        First, we give an outline of what the technique to solve a Sudoku puzzle using Gr\"obner basis involves. Then, we provide a mathematical model relating the solution to the vanishing locus of a set of polynomials generated by the constraints of the puzzle. Then, we apply it to Shidoku, a $4\times 4$ variant of Sudoku to make computations easier. The python notebook containing the code can be found \href{https://github.com/yangdabei/graph-colouring-and-sudoku/blob/main/example_shidoku.ipynb}{here}. Note that the variables in the code are 0-indexed.

        \subsection{Formulating Sudoku as a Graph Colouring Problem}
        We can formulate a Sudoku puzzle as a graph where the vertices are coloured with the numbers in the 81 squares. This gives us a graph with 81 nodes. From here on, \emph{nodes}, \emph{variables}, and \emph{cells} will be used interchangeably as they are equivalent unless otherwise stated. There is an edge between two vertices if the corresponding squares are 
        \begin{itemize}
            \item in the same row,
            \item in the same column, and 
            \item in the same $3\times 3$ block.            
        \end{itemize}
        Hence, the degree of each node is $6+6+8=20$. We use a well known formula to calculate the total number of edges for this graph.        
        \begin{theorem} \label{thm:degreesum}
            The degree sum formula states that for a given graph $G=(V,E)$,
            $$\sum_{v\in V}\deg(v)=2|E|.$$
        \end{theorem}
        By Theorem \ref{thm:degreesum}, a typical Sudoku graph has $810\times 2=810$ edges. Because of the high number of variables that naturally arise from such a large graph, we investigate and apply techniques to solve a smaller variant of Sudoku later.

        To model a Sudoku board using polynomial equations, we take inspiration from the graph polynomial defined in \ref{def:graph polynomial}, we can represent each of the cells with a different variable, say $x_1,\dots,x_{81}$, on a graph $G_S$ with 81 nodes. Every entry $a_i$ in the $i$th cell of a Sudoku board satisfies $a\in \{1,\dots,9\}$ if and only if $a_i$ is a root of the univariate polynomial $f_i(x_i)$ defined as 
        \begin{equation} \label{eq:sudoku fi}
            f_i(x_i) := \prod^9_{k=1}(x_i-k).
        \end{equation}
        Next, for $i\neq j$, the polynomial $f_i(x_i)-f_j(x_j)$ vanishes on $\mathbf{V}(x_i-x_j)$, so $x_i-x_j$ is a factor of $f_i(x_i)-f_j(x_j)$. Then, for $i\neq j$, we define the quotient of $f_i(x_i)-f_j(x_j)$ when divided by $x_i-x_j$ to be
        \begin{equation} \label{eq:sudoku hij}
            g_{ij} (x_i,x_j) = \frac{f_i-f_j}{x_i-x_j},
        \end{equation}
        where $g_{ij}\in \mathbb{Q}[x_i,x_j]$. Next, we join two vertices with an edge as described from earlier, and so the set $E$ is defined to be
        $$E = \lr\{\}{(i,j) \, | \, 1\leq i < j\leq 81, \text{$i$th and $j$th cell are in the same region}}.$$
        Let $I\subseteq\mathbb{Q}[x_1,\dots,x_{81}]$ be the ideal generated by the 891 polynomials $f_i$, $i=1,\dots,81$ and $g_{ij}$, $(i,j)\in E$. Then $I$ can be viewed as $I_{G_S,81}$ where we want to find a 9-colouring for the Sudoku graph. We will call this representation of Shidoku by the polynomial equations \ref{eq:sudoku fi} and \ref{eq:sudoku hij} the \emph{quotient Shidoku system}. Section \ref{sec:sumproduct} provides a similar method for finding a solution to a puzzle.

        \begin{proposition} \label{prop:solution}
            Let $\mathbf{V}(I)$ be the vanishing locus of $I$ in $\mathbb{Q}^{81}$, and let $a=(a_1,\dots,a_{81})$ be a point. Then $a\in \mathbf{V}(I)$ if and only if $a_i\in \{1,\dots,9\}$, for $i=1,\dots,81$, and $a_i\neq a_j$, for $(i,j)\in E$.
        \end{proposition}
        \begin{proof}
            If $a_i\in \{1,\dots,9\}$ for $i=1,\dots,81$ and and $a_i\neq a_j$, for $(i,j)\in E$, then we have already seen that $F_i$ and $H_{ij}$ vanish at $a$, so $a\in \mathbf{V}(I)$. Conversely, let $a\in \mathbf{V}(I)$. Then $F_i(a)=0$ so $a_i\in\{1,\dots,9\}$ for all $i$. Now, suppose for contradiction that $a_i=a_j=:b$ for some $(i,j)\in E$. 
            We have that 
            \begin{align*}
                g_{ij}(x_i,x_j) = \frac{f_i(x_i) -f_j(x_j)}{x_i-x_j} \implies f_i(x_i) = (x_i-x_j)g_{ij}(x_i,x_j) + f_j(x_j).
            \end{align*}
            Substituting in $b$ for $x_j$ gives $f_i(x_i)=(x_i-b)g_{ij}(x_i,b)$, and since $g_{ij}(b,b)=0$ by assumption, this implies that $b$ is a zero of $f_i$ of order at least two, which is impossible. {\color{red} CHECK PROOF}
        \end{proof}

        Next, we show that for a well-posed puzzle, the unique solution corresponds to a single point. 
        \begin{proposition} \label{prop:unique}
            Let $S$ be an explicitly given, well-posed Sudoku puzzle with preassigned numbers (clues) $\{a_i\}_{i\in L}$ for some subset $L\subset \{1,\dots,81\}$. We associate to $S$ the ideal $I_S=I+\langle\{x_i-a_i\}_{i\in L}\rangle$ (the clues of the puzzle). Then the reduced Gr\"obner basis of $I_S$ is given by $G_{red}=\langle x_1-a_1,\dots,x_{81}-a_{81}\rangle$, and $(a_1,\dots,a_{81})$ is the solution of the Sudoku.
        \end{proposition}
        \begin{proof}
            Given that $S$ is well-posed, assume that $S$ has a unique solution $(a_1,\dots,a_{81})$. By Proposition \ref{prop:solution} and the Nullstellensatz, the radical ideal $\sqrt{I_S}$ is the maximal ideal $\langle x_1-a_1,\dots,x_{81}-a_{81}\rangle$. This means that $I_S$ contains a suitable power of $x_i-a_i$ for each $i$. Since $I_S$ contains the square-free polynomials $F_i(x_i)=\prod^9_{k=1}(x_i-k)$, we get that the elimination ideals $I_S\cap K[x_i]$ are generated by the $x_i-a_i$. This, $I_S=\langle x_1-a_1,\dots,x_{81}-a_{81}\rangle$, which is the form of the reduced Gr\"obner basis.
        \end{proof}

        We now apply this property to finding the reduced Gr\"obner basis for an example Shidoku puzzle using different methods.

    \subsection{Shidoku}

            A \emph{Shidoku board} is a $4\times 4$ Latin square that whose regions (rows, columns, and designated $2\times 2$ blocks), and similarly, a \emph{well-posed Shidoku puzzle} is a partial solution to a Shidoku board that uniquely determines the board.

            \begin{figure}[h!]
                \label{fig:shidoku}
                \centering
                \begin{shidoku-block}
                    |3| | | |.
                    | |2| |4|.
                    | |1| | |.
                    | | |4| |.
                \end{shidoku-block}
                \hspace*{10pt}
                \begin{shidoku-block}
                    |3|4|1|2|.
                    |1|2|3|4|.
                    |4|1|2|3|.
                    |2|3|4|1|.
                \end{shidoku-block}
                \caption{Well-posed Shidoku puzzle and board}
            \end{figure}

            Each of the methods presented for solving the puzzle uses values from different number systems ({\color{red}?}) to represent the $n$ variables given by the $n$ cells: the roots of unity method solves for solutions in $\mathbb{C}^n$, sum-product method in $\mathbb{Z}^n$, and boolean method in $\mathbb{Z}_2^n$.

            \subsubsection{Roots of Unity Method}

            We can reformulate the techniques used to solve graph colouring in the context of sudoku. Similar to the graph colouring solution in Section 3.1, we can represent pairs of cells that share a region rather than a whole region in itself. 
            
            To start, we replace the numbers $1,2,3$ and $4$ with the fourth roots of unity, $\pm 1$ and $\pm i$. Note that we can make any arbitrary choice for the root of unity associated with a number. Then, we can encode this information in each cell $x_i$ for $1\leq i\leq 16$ which takes on values from the fourth roots of unity in 16 polynomial equations of the form
            \begin{equation} 
                \label{eq:roi}
                x_i^4-1=0.
            \end{equation}
            Next, if we fix $x_i$, for $x_j$ in the same region as $x_i$, we have another set of polynomial equations that encodes the puzzle. Since $x_i^4-x_j^4=0$, factoring gives $(x_i-x_j)(x_i+x_j)(x_i^2+x_j^2)=0$. Now, $x_j$ cannot be the same number as $x_i$ because they must be distinct roots of unity so $x_i-x_j\neq0$. Hence, we also have polynomials of the form
            \begin{equation}
                \label{eq:roi factor}
                (x_i+x_j)(x_i^2+x_j^2)=0.
            \end{equation}
            The 56 polynomials of this form, along with the 16 from Equation \ref{eq:roi}, gives 72 polynomials total that we can generate our ideal $I_{RoI}$. Lastly, we add in the constraints given by the clues. If we set the symbols $\{1,2,3,4\}$ to the complex fourth roots of unity $\{i,-1,-i,1\}$, we also add the equations $x_1+i=0, x_6+1=0, x_8-1=0, x_{10} -i = 0, x_{15}-1=0$ to the ideal. We can then check for the solution of the corresponding variety.

            \subsubsection{Sum-Product Method} \label{sec:sumproduct}

            We can also have a representation of the board based on its regions. Every cell can take on a number from $\{1,2,3,4\}$. Fix $i$ for $1\leq i\leq 16$. For each $x_i$, we can encode this in the polynomial equation
            \begin{equation}
                \label{eq:sp null}
                (x_i-1)(x_i-2)(x_i-3)(x_i-4)=0
            \end{equation}
            since the $x_i$ must be one of $\{1,2,3,4\}$. Now suppose we have four cells $w,x,y,z$ in the same region. It turns out that the only way to choose four numbers that sum to 10 and multiply to 24 from the set $\{1,2,3,4\}$ is to use each number once. (Note that : {\color{red} unique set of numbers for arbitrary $n\times n$.} For example, in a normal Sudoku, there is more than one choice of selecting numbers from $\{1,2,3,4,5,6,7,8,9\}$ that sum to $45$ and multiply to $9!=362880$, namely $\{1,2,3,4,5,6,7,8,9\}$ and $\{1,2,4,4,4,5,7,9,9\}$. We can instead assign each cell a number from $\{-2,-1,1,2,3,4,5,6,7\}$ since it is the smallest set in magnitude for which each of the nine elements are picked exactly once to make the sum and product). This means that for $w,x,y,z$ in each region, we have polynomial equations of the form
            \begin{equation}
                \label{eq:sp sum}
                w+x+y+z-10=0, \text{ and} 
            \end{equation}
            \begin{equation}
                \label{eq:sp prod}
                \quad wxyz -24 = 0.
            \end{equation}
            We get 16 equations from Equation \ref{eq:sp null} and 24 equations from Equation \ref{eq:sp sum} and Equation \ref{eq:sp prod} since there are 12 regions, giving a total of 40 initial polynomials that we can use to generate the ideal $I_{SP}$. We also add constraints based on the cells that have been filled on any given Shidoku puzzle. For example, if we consider the Shidoku puzzle in Figure \ref{fig:shidoku} (left), we would also add $x_1-3=0$, $x_6-2=0$, $x_8-4=0$, $x_{10}-1=0$, and $x_{15}-4=0$.  

            \subsubsection{Boolean Variable Method}

            Lastly, we explain the Boolean method. For each cell, we introduce four variables for each cell $x_{i,1}, x_{i,2}, x_{i,3}$, and $x_{i,4}$ for $1\leq i\leq 16$ (note that now {color{red} dfdfd}), where we set $x_{i,k}=1$ when cell $x_i$ takes the value $k$, and $x_{i,k}=0$ otherwise. Encoding the individual cells for the puzzle now takes 64 variables instead of 16 (it is suggested by \citet{bernasconi1997computing} and \citet{sato2008computation} that the cost of finding a Gr\"obner basis is greatly reduced). For each $i$, we then get polynomials of the form
            \begin{equation} \label{eq:bool}
                x_{i,k}(x_{i,k}-1)=0
            \end{equation}

            Because each $x_{i,j}$ can only take on values $0$ and $1$, we also get 16 polynomials of the form
            \begin{equation} \label{eq:bool sum}
                x_{i,1}+x_{i,2}+x_{i,3}+x_{i,4}-1=0
            \end{equation}
            Finally, we require any two cells $x_i$ and $x_j$ in the same region to have different values. Therefore, for each possible $k$, at least one of $x_{i,k}$ and $x_{j,k}$ must be 0. We have 56 polynomial equations of the form
            \begin{equation} \label{eq:bool prod}
                x_{i,1}x_{j,1} + x_{i,2}x_{j,2} + x_{i,3}x_{j,3} + x_{i,4}x_{j,4} = 0.
            \end{equation}
            We get a total of 136 initial polynomials that we can use to generate our ideal $I_{BV}$. Lastly, we need to add the equations $x_{1,3}-1, x_{6,2}-1, x_{8,4}, x_{10, 1}$, and $x_{15,4}-1$ to the ideal. We then find the reduced Gr\"obner basis to find the solution.

        
    \subsection{Kropki Sudoku}

    Whilst the Gr\"obner basis method might be overshadowed by classical Sudoku solving heuristics, it can provide a solution to variants of Sudoku. We look at a \href{https://www.funwithpuzzles.com/2018/02/kropki-or-dots--sudoku-puzzle.html}{Sudoku with Kropki dots}. The normal rules of Sudoku apply with two added constraints: if the absolute difference between any two digits in neighbouring cells is 1, the cells are separated by a white dot, and if the digit is half of the digit in the neighbouring cell, they are separated by a black dot. We can model the Kropki dots using polynomial equations. If cells $x_i$ and $x_j$ are separated by a white dot, then we get polynomials of the form
    \begin{equation} \label{eq:kropki white}
        (x_i-x_j)(x_j-x_i)+1=0
    \end{equation}
    since either $x_i-x_j=-1$ or $x_j-x_i=1$ and the other giving 1. Now, if $x_i$ and $x_j$ are separated by a black dot, we can model this with polynomials of the form
    \begin{equation} \label{eq:kropki black}
        (x_i-2x_j)(2x_i-x_j) = 0.
    \end{equation}
    This is because we are not sure which one of $x_i$ or $x_j$ is double the other. We get that either $x_i-2x_j=0$ or $2x_i-x_j$. Along with the polynomials generated from the usual constraints of Sudoku, we add Equations \ref{eq:kropki white} and \ref{eq:kropki black} to our ideal and compute a Gr\"obner basis.
    
    However, using the \texttt{sympy} package in Python is very inefficient and is not feasible to find a Gr\"obner basis and solve for over 900 polynomials. We provide the code 

\end{document}
        