\documentclass[../main.tex]{subfiles}

\begin{document}


    \subsection*{Key Definitions}
    A \emph{graph} is an ordered pair $G=(V,E)$, which consists of a nonempty set $V$ of \emph{nodes} and a set $E$ of paired vertices whose elements are called \emph{edges} 

    Let $G=(V,E)$ be an $n$-node graph. For a node $v\in V$, the \emph{neighbourhood} of $v$ is given by $N(v)=\{u\in V\, |\, \{u,v\}\in E\}$. 

    \begin{theorem} \label{thm:degreesum}
        The degree sum formula states that for a given graph $G=(V,E)$,
        $$\sum_{v\in V}\deg(v)=2|E|.$$
    \end{theorem}

    \begin{definition}
        For a graph $G=(V,E)$, a \emph{colouring} is a function $\mathcal{C}:V\to \{1,2,\dots\}$ such that for all $u\in N(v)$, $\mathcal{C}(u)\neq\mathcal{C}(v)$. $G$ is $k$-colourable if $G$ can be coloured with $k$ distinct colours. 
    \end{definition}
    



    Note that we use the terminology \emph{colours} for node labels and labels such as \emph{red} and \emph{blue} are used when the number of colours are small.



    \begin{definition}
        The \emph{graph polynomial} $f_G$ associated to the graph $G=(V,E)$ is an element of the ring $\mathbb{C}[x_1,\dots,x_n]$, given by:
        $$f_G := \prod_{\{u,v\}\in E}(u-v)$$
    \end{definition}



    \begin{theorem}[Hilbert's Nullstellensatz]
        Let $I\subseteq \mathbb{C}[k_1,\dots,x_n]$. Then $f\in\mathbf{I}(\mathbf{V}(I))$ if and only if there exists some integer $m$ such that $f^m\in I$.
    \end{theorem}

    %\subsection{Roots of Unity}

    \begin{theorem}
        Fix $k$ a positive integer. Let I be the ideal generated by the polynomials $v^k-1$ for $v\in V$. The graph $G$ is $k$-colourable if and only if $f_G\notin I$.
        \begin{proof}
            Let $G$ be $k$-colourable. Then there is an assignment of colours to the vertices such that no two adjacent vertices have the same colour. This corresponds to a point $a\in \mathbf{V}(I)$ such that $f_G(a)\neq 0$. Hence, $f_G\notin I$.

            Conversely, if $G$ is not $k$-colourable, then there is at elast one pair of adjacent vertices that share a colour. This means that $f$ vanishes for any assignment of colours, that is, $f$ vanishes on $\mathbf{V}(I)$. By Hilbert's Nullstellensatz, there is some $m$ such that $f^m\in I$. Since $I$ is a radical ideal ({\color{red} PROVE})
        \end{proof}
    \end{theorem}

    For each $v\in V$, the polynomials that generate $I$ represent the $k$-th roots of unity

    We can also equivalently formulate the question as follows. Let $x=e^{\frac{2\pi i}{k}}\in \mathbb{C}$ be the $k$-th root of unity. Represent the $k$ colours by the $k$ distinct roots of unity, so each node is assigned $1, x, x^2,\dots,x^{k-1}$. We can model this as
    \begin{equation} \label{eq:k_roots}
        x_i^k-1=0,\, 1\leq i\leq n
    \end{equation}

    if $x_i$ and $x_j$ are connected by an edge, they need to be a different colour. Since $x_i^k = x_j^k$, we have that $(x_i-x_j)(x_i^{k-1} + x_i^{k-2}x_j+\dots+x_ix_j^{k-2}+x_j^{k-1})$. We require $x_i$ and $x_j$ to be different $k$-th roots of unity so 
    \begin{equation} \label{eq:edges}
        x_i^{k-1} + x_i^{k-2}x_j+\dots+x_ix_j^{k-2}+x_j^{k-1}=0.
    \end{equation}


    Let the ideal $I$ be generated by the polynomials in Equation \ref{eq:k_roots} and for each pair of adjacent vertices $x_i, x_j$ y the polynomials in Equation \ref{eq:edges}

    The following definition relates these polynomial equations to an ideal that we can analyse.

    \begin{definition}
        The $k$-colouring of an $n$-node graph $G$ is the ideal $I_{G,k}\subseteq \mathbb{C}[x_1,\dots,x_n]$ generated by 
        \begin{align*}
            \text{for all $i\in V(G)$}&: \quad x_i^k-1 \\
            \text{for all $\{i,j\}\in E(G)$}&: \quad x_i^{k-1} + x_i^{k-2}x_j+\dots+x_ix_j^{k-2}+x_j^{k-1}
        \end{align*}
    \end{definition}

    \subsection{3-Colouring}

    For $k=3$, the ideal $I_{G,3}$ is generated by the following polynomials:
    \begin{align*}
        \text{for all $i\in V(G)$}&:\quad x_i^3-1 \\
        \text{for all $ij\in E(G)$}&:\quad x_i^2+x_ix_j+x_j^2
    \end{align*}

    The 3-colouring probem is known to be NP-Complete.

    \subsection{The Chromatic Number}

    Another interesting question that arises from the colouring problem is what the smallest number of colours required to colour a graph. We give the 

    \begin{definition}
        The \emph{chromatic number} $\chi=\chi(G)$ is the smallest number such that the graph can be coloured with $\chi$ colours. A $k$-colourable graph is $k$-chromatic if its chromatic number is $k$.
    \end{definition}

    We first give a few definitions.

    \begin{definition}[Independent Set, Vertex Cover]
        For a graph $G=(V,E)$, an \emph{independent set} is a set of vertices $U\subseteq V$ such that there are no edges between any two vertices in $U$. The independence number $\alpha=\alpha(G)$ is the size of the largest independent set. A subset $W\subseteq V$ is a \emph{vertex cover} such that for all $\{u,v\}\in E$, $u\in W$ or $v\in W$. 
    \end{definition}

    We see that a vertex cover of a graph is the complement of an independent set. Hence, a maximal independent set corresponds to a minimal vertex cover. Furthermore, each node in an independent set can be coloured the same colour.

    {\color{red} INSERT EXAMPLE}

    \begin{definition}[Cover Ideal]
        For a graph $G$, the \emph{colour ideal} $I_{cover}(G)$ is the monomial ideal
        $$I_{cover}(G) := \bigcap_{\{u,v\}\in E}(u)$$
    \end{definition}
\end{document}