\documentclass[../main.tex]{subfiles}

\begin{document}

    \subsection{Monomial Orders and the Division Algorithm}

    We wish to extend the notion of the division algorithm from $k[x]$ to $k[x_1,\dots,x_n]$. However, it is not so trivial. We will see the complications that arise if we try extending it na\"ively in the following section.

    \begin{theorem}[Division Algorithm]
        Let $k$ be a field and let $g$ be a non-zero polynomial in $k[x]$. Then ever $f\in k[x]$ can be written as $$f=qg+r,$$ where $q,r\in k[x]$, and either $r=0$ or $\deg(r)<\deg(g)$. Furthermore, $q$ and $r$ are unique, and there is an algorithm for finding $q$ and $r$.
    \end{theorem}
    We also have a notion of a greatest common divisor for two polynomials.
    \begin{definition}
        A \emph{greatest common divisor} of polynomials $f,g\in k[x]$ is a polynomial $h$ such that:
        \begin{enumerate}[(i)]
            \item $h$ divides $f$ and $g$.
            \item If $p$ is another polynomial which divides $f$ and $g$, then $p$ divides $h$. When $h$ has these properties, we write $h=\gcd(f,g)$.
        \end{enumerate}
    \end{definition}
    Using the division algorithm, we have a way of computing the greatest common divisor of two polynomials.
    \begin{theorem}[Euclidean Algorithm]
        For $f,g\in k[x]$, $g\neq0$, we have that $\langle f,g\rangle =\langle \gcd(f,g)\rangle$. We can also explicitly find $\gcd(f,g)$, which is the last non-zero remainder $r_n$ in the sequence of divisions
        \begin{align*}
            f&=gq_1+r_1 \\
            g &= r_1q_2+r_2\\
            r_1 &= r_2q_3+r_3 \\
            &\vdots \\
            r_{n-2}&=r_{n-1}q_n+r_n \\
            r_{n-1} &= r_nq_{n+1}+0.
        \end{align*}
    \end{theorem}
    Using these algorithms, we can analyse the structure of ideals. Questions that arise include the problem of \emph{ideal membership}, i.e., is $f\in \langle f_1,\dots,f_s\rangle$, and \emph{ideal equality}, i.e., does $\langle f_1,\dots,f_s\rangle = \langle g_1,\dots,g_t\rangle$. In the single variable case, we determined the ``size'' of a polynomial by its degree. This means that the algorithms would eventually terminate as at each stage, the polynomials being produced would have a smaller degree. We now extend this idea of ``size'' to multivariable polynomials.

    \begin{definition}
        A \emph{monomial} in $x_1,\dots,x_n$ is a product of the form
        $$x_1^{\alpha_1}\cdot\dots\cdot x_n^{\alpha_n},$$
        where all of the exponents $\alpha_1,\dots,\alpha_n$ are non-zero
    \end{definition}


    \begin{definition}
        A \emph{monomial ordering} $\succ$ on $k[x_1,\dots,x_n]$ is a relation on the set of monomials $x^\alpha, \alpha\in\mathbb{Z}^n_{\geq 0}$ (i.e., the monomial exponents), such that:
        \begin{enumerate}
            \item The relation $\succ$ is a total (or linear) ordering.
            \item If $\alpha \succ \beta$ and $\gamma\in \mathbb{Z}^n_{\geq0}$, then $\alpha+\gamma\succ\beta+\gamma$.
            \item The relation $\succ$ is a well-ordering on $\mathbb{Z}^n_{\geq0}$ (every non-empty subset has a smallest element).
        \end{enumerate}
    \end{definition}
    With respect to a monomial ordering, we can compare the ``sizes'' of monomials which means we can define a systematic way of division with multivariate polynomials. Here are two basic examples of monomial orderings:
    \begin{itemize}
        \item (lexicographic order) We say $\alpha\succ\beta$ if the left-most non-zero entry of $\alpha-\beta$ is positive.
        \item (graded lexicographic order) We say $\alpha\succ\beta$ if $|\alpha|\succ|\beta|$, or $|\alpha|=|\beta|$ and $\alpha\succ_{lex}\beta$.
    \end{itemize}
    %In the later sections, we will be mostly using lexicographic ordering.
    

    %\begin{definition}
    %    For a given monomial order $\succ$ and a non-zero %polynomial $f=\sum_\alpha a_\alpha x^\alpha$, we give %the following definitions:
    %    \begin{itemize}
    %        \item The \emph{multidegree} of $f$ is $$\text%{multideg}(f) = \max(\alpha\in\mathbb{Z}^n_{\geq0}%\,|\,a_\alpha\neq0)$$ or the maximum exponent %among $\alpha$ with respect to $\succ$.
    %        \item The \emph{leading coefficient} of $f$ is %$$LC(f) = a_{\text{multideg$(f)$}\in k}.$$
    %        \item The \emph{leading monomial} of $f$ is $$LM%(f)=x^\text{multideg$(f)$}$$ with coefficient 1.
    %        \item The \emph{leading term} of $f$ is $$LT(f) = %LC(f)\cdot LM(f).$$
    %    \end{itemize}
    %\end{definition}    

    Now for $f\in k[x_1,\dots,x_n]$, let $LT(f)$ denote the leading term under a monomial ordering $\succ$. Then, we formulate the division algorithm for multivariate polynomials.
    \begin{theorem}[Multivariate Division Algorithm]
        Let $\succ$ be a monomial ordering and let $f_1,\dots,f_s,\in k[x_1,\dots,x_n]$ be. Then every $f\in k[x_1,\dots,x_n]$ can be written as 
        $$f=q_1f_1+\dots q_sf_s+r,$$
        where $q_i,r\in k[x_1,\dots,x_n]$ where no $LT(f_i)$ is divisible by any term in the remainder $r$. 
    \end{theorem}
    
    \subsection{Definition and Properties of Gr\"obner Bases}
    The ideal membership problem is difficult to determine if we only consider using the division algorithm. The next example demonstrates that we cannot determine whether $f$ is in $\langle f_1,\dots,f_s\rangle$ simply by dividing $f$ with the generators.

    \begin{example} \label{ex:div}
        Consider $k[x,y,z]$ with lex order. Let $I=\langle g_1,g_2\rangle$ where $g_1=x^2y-z$ and $g_2=xy-1$, and let $f=x^3-x^2y-x^2z+x$. We see that $f\in I$ since
        $$f=x^2\cdot g_1+(-x^3-x)\cdot g_2.$$
        However, using polynomial long division to divide $f$ by $g_1$ gives a remainder of $x^3-x^2z+x-z$, which is not divisible by $g_2$. Likewise, dividing by $g_2$ gives a remainder of $x^3-x^2z$, which is not divisible by $g_1$.
    \end{example}

    This problem can be solved if we find a special generating set that is finite for the ideal $I$. Such a generating set is known as a \emph{Gr\"obner basis}. We use the Hilbert Basis Theorem to guarantee that a Gr\"obner basis always exists for any ideal.
    \begin{theorem}[Hilbert Basis Theorem]
        Every polynomial ideal $I\subseteq k[x_1,\dots,x_n]$ is finitely generated.
    \end{theorem}
    
    \begin{definition}
        Let $I\subseteq k[x_1,\dots,x_n]$ be an ideal. Given a monomial ordering $\succ$, a finite subset $G=\{g_1,\dots, g_t\}\subseteq I$ different from $\{0\}$ is said to be a \emph{Gr\"obner basis} if
        \begin{enumerate}
            \item $G$ generates $I$, and
            \item $\langle LT(g_1),\dots,LT(g_t)\rangle=\langle LT(I)\rangle$, where $LT(I)$ is the set of leading terms of non-zero elements of $I$.
        \end{enumerate}
    \end{definition}

    How does this solve our problem of checking ideal membership? We will use the following proposition.
    \begin{proposition}
        Let $I\subseteq k[x_1,\dots,x_n]$ be an ideal and $G=\{g_1,\dots,g_t\}$ be a Gr\"obner basis for $I$. Then given $f\in k[x_1,\dots,x_n]$, there is a unique $r\in k[x_1,\dots,x_n]$ such that 
        \begin{enumerate}[(i)]
            \item no term of $r$ is divisible by any of $LT(g_1),\dots,LT(g_t)$, and
            \item there is a $g\in I$ such that $f=g+r$.
        \end{enumerate}
    \end{proposition}
    Restated, this proposition says that $f\in I$ if and only if $f$ vanishes when divided by a Gr\"obner basis for $I$.

    \begin{example}
        From Example \ref{ex:div}, a Gr\"obner basis for the ideal $I=\langle g_1,g_2\rangle$ is given by $G=\{yz-1,x-z\}$. Note that $yz-1=g_1-x\cdot g_2$ and $x-z=-y\cdot g_1+(1+xy)\cdot g_2$, so $I\subseteq \langle yz-1,x-z\rangle$. Also, $g_1 = x\cdot(yz-1)+(1+xy)\cdot(x-z)$ and $g_2=(yz-1)+y\cdot(x-z)$, hence, $\langle yz-1,x-z\rangle = \langle g_1,g_2\rangle$. Furthermore, it can be checked that $\langle LT(yz-1,x-z)\rangle = \langle LT(yz-1), LT(x-z)\rangle$, so it is indeed a Gr\"obner basis. Since long dividing $f$ by $yz-1$ and $x-z$, we find that the remainder is zero, so $f\in I$.
    \end{example}

    %Computing a Gr\"obner basis 
%
    %\begin{definition}
    %    A \emph{reduced Gr\"obner basis} for a polynomial %ideal $I$ is a Gr\"obner basis $G$ for $I$ such that:
    %    \begin{enumerate}[(i)]
    %        \item The leading coefficient of $g$ is equal to %1 for all $g\in G$.
    %        \item For all $g\in G$, no monomial of $g$ lies %in $\langle LT(G\setminus\{g\})\rangle$.
    %    \end{enumerate}
    %\end{definition}

    There is also the notion of a minimal Gr\"obner basis. We call this a \emph{reduced Gr\"obner basis}.

    \begin{definition}
        A \emph{reduced Gr\"obner basis} for a polynomial ideal $I$ is a Gr\"obner basis $G$ for $I$ such that:
        \begin{enumerate}[(i)]
            \item The leading coefficient of $p$ is 1 for all $p\in G$.
            \item For all $p\in G$, no monomial of $p$ lies in $\langle LT(G\setminus\{p\})\rangle$.
        \end{enumerate}
    \end{definition}

    There are many algorithms to determine a Gr\"obner basis for an ideal such as Buchberger's algorithm (refer to Chapter 2 $\mathsection 7$ of \cite{cox2013ideals}), however they are quite technical and outside the scope of this report.

\end{document}