\documentclass[../main.tex]{subfiles}

\begin{document}


    Finding a special generating set for the ideal $I$. For $f\in $

    \begin{definition}
        A \emph{monomial} in $x_1,\dots,x_n$ is a product of the form
        $$x_1^{\alpha_1}\cdot\dots\cdot x_n^{\alpha_n},$$
        where all of the exponents $\alpha_1,\dots,\alpha_n$ are non-zero
    \end{definition}

    \begin{definition}
        A \emph{monomial ordering} $\succ$ on $k[x_1,\dots,x_n]$ is a relation on the set of monomials $x^\alpha, \alpha\in\mathbb{Z}^n_{\geq 0}$ (i.e., the monomial exponents), such that:
        \begin{enumerate}
            \item The relation $\succ$ is a total (or linear) ordering.
            \item If $\alpha \succ \beta$ and $\gamma\in \mathbb{Z}^n_{\geq0}$, then $\alpha+\gamma\succ\beta+\gamma$.
            \item The relation $\succ$ is a well-ordering on $\mathbb{Z}^n_{\geq0}$ (every non-empty subset has a smallest element)
        \end{enumerate}
    \end{definition}

    \begin{definition}
        A \emph{monomial ideal} is a polynomial ideal generated by monomials.
    \end{definition}

    \begin{theorem}[Division Algorithm]
        Let $\succ$ be a monomial ordering and let $f_1,\dots,f_s,\in k[x_1,\dots,x_n]$ be. Then every $f\in k[x_1,\dots,x_n]$ can be written as 
        $$f=q_1f_1+\dots q_sf_s+r,$$
        where $q_i,r\in k[x_1,\dots,x_n]$ where no $LT(f_i)$ is divisible by any term of the remainder $r$. 


    \end{theorem}

    \begin{theorem}[Hilbert Basis Theorem]
        Every polynomial ideal $I\subseteq k[x_1,\dots,x_n]$ is finitely generated.
    \end{theorem}

    \begin{definition}[Gr\"obner Basis]
        Let $I\subseteq k[x_1,\dots,x_n]$ be an ideal. Given a monomial ordering $<$, a finite subset $G=\{g_1,\dots, g_t\}\subseteq I$ different from $\{0\}$ is said to be a \textbf{Gr\"obner basis} if
        \begin{enumerate}
            \item $G$ generates $I$, and
            \item $\langle LT(g_1),\dots,LT(g_t)\rangle=\langle LT(I)\rangle$, where $LT(I)$ is the set of leading terms of non-zero elements of $I$.
        \end{enumerate}
    \end{definition}



    \begin{theorem}
        
    \end{theorem}

\end{document}