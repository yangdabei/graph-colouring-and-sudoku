\documentclass[../main.tex]{subfiles}

\begin{document}

   A Gr\"obner basis is a specific kind of generating set of an ideal in a polynomial ring. It is one of the main practical tools used in computer algebra, computational algebraic geometry, and computational commutative algebra for solving systems of polynomial equations as well as finding the images of algebraic varieties under projections or rotational maps. Introduced in 1965 by Bruno Buchberger in his Ph.D. thesis \cite{buchberger2006bruno}, Buchberger determined an algorithm to find such a basis. In this report, we will discuss the basic properties of Gr\"obner bases. We also present and implement different methods involving Gr\"obner bases to problems in graph colouring as well as extending this to solving Sudoku puzzles and its variants. We first give some preliminary definitions and theorems from \cite{cox2013ideals}.

   \begin{definition}
      A subset $I\subseteq k[x_1,\dots,x_n]$, where $k$ is a field, is an \emph{ideal} if it satisfies:
      \begin{enumerate}
         \item $0\in I$.
         \item If $f,g\in I$, then $f+g\in I$.
         \item If $f\in I$ and $h \in k[x_1,\dots,x_n]$, then $hf\in I$.
      \end{enumerate}
   \end{definition}

   \begin{lemma}
      If $f_1,\dots, f_s\in k[x_1,\dots,x_n]$, then 
      $$\langle f_1,\dots,f_s\rangle = \lr\{\}{\sum^s_{i=1}h_if_i\,|\,h_1,\dots,h_s\in k[x_1,\dots,x_n]}$$ 
      is an ideal of $k[x_1,\dots,x_n]$.
   \end{lemma}
   We will call $\langle f_1,\dots,f_s\rangle$ the \emph{ideal generated by} $f_1,\dots,f_s$.
   \begin{definition}
      If $I$ and $J$ are two ideals in $k[x_1,\dots,x_n]$, then their \emph{product}, denoted by $I\cdot J$, is defined to be the ideal generated by all polynomials $f\cdot g$ where $f\in I$ and $g\in J$. Thus, the product $I\cdot J$ is the set
      $$I\cdot J=\{f_1g_1+\dots+f_rg_r\,|\,f_1,\dots,f_r\in I, g_1,\dots,g_r\in J, r \text{ a positive integer}\}.$$
   \end{definition}
   \begin{definition}
      Let $f_1,\dots,f_s$ be polynomials in $k[x_1,\dots,x_n]$. Then we set
      $$\mathbf{V}(f_1,\dots,f_s) = \lr\{\}{(a_1,\dots,a_n)\in k^n \,|\, f_i(a_1,\dots,a_n)=0 \text{ for all $1\leq i\leq s$}}.$$
   \end{definition}
   We will call $\mathbf{V}(f_1,\dots,f_s)$ the \emph{affine variety} defined by $f_1,\dots,f_s$.
   \begin{lemma}
      If $f_1,\dots,f_s$ and $g_1,\dots,g_t$ are generators of the same ideal in $k[x_1,\dots,x_n]$, then we have $\mathbf{V}(f_1,\dots,f_s)=\mathbf{V}(g_1,\dots,g_t)$.
   \end{lemma}
   \begin{definition}
      An affine variety $\mathbf{V}(I)$ where $I=\langle f_1,\dots,f_s\rangle$ is defined by
      $$\mathbf{V}(I)=\mathbf{V}(f_1,\dots,f_s) = \lr\{\}{(a_1,\dots,a_n)\in k^n \, |\, f_i(a_1,\dots,a_n)=0 \text{ for all $1\leq i\leq s$}}.$$
   \end{definition}
   \begin{definition}
      Let $V\subseteq k^n$ be an affine variety. Then we set
      $$\mathbf{I}(V)=\{f\in k[x_1,\dots,x_n] \, |\, f(a_1,\dots,a_n)=0 \text{ for all $(a_1,\dots,a_n)\in V$}\}.$$
   \end{definition}

   \begin{theorem}[The Weak Nullstellensatz]
      Let $k$ be an algebraically closed field and let $I\subseteq k[x_1,\dots,x_n]$ be an ideal satisifying $\mathbf{V}(I)=\emptyset$. Then $I=k[x_1,\dots,x_n]$.
   \end{theorem}

   \begin{theorem}[Hilbert's Nullstellensatz]
      Let $I\subseteq \mathbb{C}[k_1,\dots,x_n]$. Then $f\in\mathbf{I}(\mathbf{V}(I))$ if and only if there exists some integer $m$ such that $f^m\in I$.
  \end{theorem}


   Throughout the report, we give the sample code for each problem. The repository for this project can be found \href{https://github.com/yangdabei/graph-colouring-and-sudoku}{here}. The computations will be done in Python using the \href{https://www.sympy.org/en/index.html}{\texttt{sympy}} package as it is easier to work with graphs by using the \href{https://networkx.org/}{\texttt{networkx}} package. However, note that using Mathematica is vastly superior in terms of its computational speed for computer algebra.

\end{document}